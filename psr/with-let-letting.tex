
This PSR adds defintions for \gn{let\_clause} and \gn{letting\_clause} to figure 3 of the PartiQL specification 
and include them as optional clauses for \gl{SELECT} queries.

\newcounter{rownum_ctr}
\newcommand\row{\stepcounter{rownum_ctr}\arabic{rownum_ctr}}

\begin{figure}[ht!]
\centering
\scriptsize
\begin{tabular}{|@{~}rc@{~}l@{~}|}
\hline
        \row & \multicolumn{2}{@{}l@{~}|}{\gn{query}} \\
        \row & \gp & \gn{sfw\_query} \\
        \row & \gd & \gn{expr\_query} \\ 

        \hline

        \row & \multicolumn{2}{@{}l@{~}|}{\gn{sfw\_query}} \\
        \row & \gp & (\gt{WITH} \gn{query} \gt{AS} \gn{variable})? \\
        \row &     & \gn{select\_clause} \\ %FIXME (see Figure~\ref{figure:select:bnf}) \\
        \row &     & \gn{from\_clause} \\   %FIXME (see Figure~\ref{figure:from:bnf}) \\ 
        \row &     & \textbf{(\gn{let\_clause)}? $\leftarrow $ NEW}   \\
        \row &     & (\gt{WHERE} \gn{expr\_query})? \\
        \row &     & (\gt{GROUP BY} \gn{expr\_query} (\gt{AS} \gn{variable})? \\ 
        \row &     & ~~~~(\gt{,} \gn{expr\_query} (\gt{AS} \gn{variable})?)*)? \\
        \row &     & ~~~~\gt{GROUP AS} \gn{variable} \\
        \row &     & (\gt{HAVING} \gn{expr\_query})? \\
        \row &     & \textbf{(\gt{LETTING} \gn{expr\_query})? $\leftarrow$ NEW} \\
        \row &     & ((\gt{OUTER})? (\gt{UNION}\gd\gt{INTERSECT}\gd\gt{EXCEPT}) \gt{ALL}? \gn{sfw\_query})? \\
        \row &     & ((\gt{ORDER BY} (\gn{expr\_query} (\gt{ASC}\gd\gt{DESC})? \gs{order\_spec}?  \\
        \row &     & ~~~~~~~~(\gt{,} \gn{expr\_query} (\gt{ASC}\gd\gt{DESC})? \gs{order\_spec}?)*)? ) \\
        \row &     & ~~~~| \gt{PRESERVE})? \\
        \row &     & (\gt{LIMIT} \gn{expr\_query})? \\
        \row &     & (\gt{OFFSET} \gn{expr\_query})? \\ 
        
        \hline

        \row & \multicolumn{2}{@{}l@{~}|}{\gn{expr\_query}} \\
        \row & \gp & \gt{(} \gn{sfw\_query} \gt{)} \\
        \row & \gd & \gn{path\_expr} \\
        \row & \gd & \gn{function\_name} \gl{(} (\gn{expr\_query} (\gt{,} \gn{expr\_query})*)? \gl{)} \\
        \row & \gd & \gt{\{} (\gs{expr\_query}\gt{:}\gn{expr\_query} (\gt{,} \gs{expr\_query}\gt{:}\gn{expr\_query})*)? \gt{\}} \\
        \row & \gd & \gt{[} (\gn{expr\_query} (\gt{,} \gn{expr\_query})*)? \gt{]} \\
        \row & \gd & \gl{\ob} (\gn{expr\_query} (\gt{,} \gn{expr\_query})*)? \gl{\cb} \\
        \row & \gd & \gs{sql\_scalar\_expr} \\
        \row & \gd & \gs{value\_constant} \\ 
        
        \hline

        \row & \multicolumn{2}{@{}l@{~}|}{\gn{path\_expr}} \\
        \row & \gp & \gs{variable} \\
%       \row & \gd & \gt{(} \gn{sfw\_query} \gt{)} \\
        \row & \gd & \gt{(} \gn{expr\_query} \gt{)} \\
        \row & \gd & \gn{path\_expr} \gt{.} \gs{attr\_name} \\
        \row & \gd & \gn{path\_expr} \gt{[} \gs{expr\_query} \gt{]} \\
        \row & \gd & \gn{path\_expr} \gt{.} \gl{*} \\
        \row & \gd & \gn{path\_expr} \gt{[} \gl{*} \gt{]} \\      

        \hline
        \row & \textbf{$\downarrow$ NEW} & \\
        \row & \multicolumn{2}{@{}l@{~}|}{\textbf{\gn{let\_clause}}} \\
    
        \row & \gp & \textbf{\gt{LET} \gn{expr\_query} \gt{AS} \gn{variable} (, \gn{expr\_query} \gt{AS} \gn{variable})*} \\

        \hline
        \row & \textbf{$\downarrow$ NEW} & \\
        \row & \multicolumn{2}{@{}l@{~}|}{\textbf{\gn{letting\_clause}}} \\
    
        \row & \gp & \textbf{\gt{LETTING} \gn{expr\_query} \gt{AS} \gn{variable} (, \gn{expr\_query} \gt{AS} \gn{variable})*} \\
        
\hline
\end{tabular}
\caption{Figure 3 of the PartiQL specification.}
\label{figure:query:bnf}
\end{figure}

\section{\gl{WITH}, \gl{LET} and \gl{LETTING}} 
\label{sec:with-let-letting}
This PSR also adds the following section to the PartiQL specificaiton.

\begin{figure}[ht!]
    \centering
    \begin{tabular}{lll}
        \gn{var\_decl} & \gp & \gn{expr\_query} \gt{AS} \gn{binding} \\ 
        \gn{var\_decl\_list} & \gp & \gn{var\_decl} (, \gn{var\_decl})* \\
        \gn{with\_clause} & \gp & (to do) \\
        \gn{let\_clause} & \gp & \gt{LET} \gn{var\_decl\_list} \\
        \gn{letting\_clause} & \gp & \gt{LETTING} \gn{var\_decl\_list} \\
    \end{tabular}
    \caption{BNF Grammar for \gl{WITH}, \gl{LET} and \gl{LETTING}}
    \label{figure:let}
\end{figure}

\gl{WITH}, \gl{LET} and \gl{LETTING} allow the user to bind expressions to variable names.  
Their purpose and semantics are similar, with differences in the context in
which they execute and the scope of the variables they define.  Figure
\ref*{figure:query:bnf} shows the contexts in which these clauses are allowed.

\subsection*{Lexical Scoping}

Every \gn{var\_decl} included with \gl{WITH}, \gl{LET} or \gl{LETTING} is
establishes a new environment which is nested according to the lexical order of
the \gn{var\_decl\_list}.  The \emph{last} nested environment is then accessible
in different parts of a \gl{SELECT} query for each \gl{WITH}, \gl{LET} and
\gl{LETTING} clauses as described in the following sections.

To illustrate the lexcial scoping rules of a \gn{var\_decl\_list} we will use
the \gl{LET} clause, however these rules are the same for \gl{WITH} and
\gl{LETTING}.

\bigskip

\noindent Variables defined later in the \gn{var\_decl\_list} \emph{shadow} any
preceding variables with the same name.  The value of \lstinline{x} is
\lstinline{42}:

\begin{lstlisting}
LET 1 AS x, 42 AS x 
\end{lstlisting}

\noindent Furthermore, shadowing may also occur with variables defined
\emph{prior} to the \gl{LET} clause.  The result of the
following query is \lstinline{<< 42 >>}.

\begin{lstlisting}
SELECT VALUE x FROM << { 'y': 42 } >> AS x LET x = x.y
\end{lstlisting} 

\noindent Variables defined earlier \gn{var\_decl\_list} cannot access variables defined
later.  Here, an error occurs inside the definition of \lstinline{y} because
\lstinline{x} has not been defined yet:

\begin{lstlisting}
    LET x + 1 AS y, 41 AS x 
\end{lstlisting}

\subsection{WITH clause}

The \gl{WITH} clause defines variables to be defined \emph{before} query
evaluation. Variables defined using \gl{WITH} are scoped to the associated
\gn{expr\_query}.

\bigskip

TODO...

\subsection{LET clause} 
\label{sec:let-clause}

\gl{LET} is an optional clause of a \gl{SELECT} query that immediately follows
the \gl{FROM} clause. \gl{LET} executes once for every environment produced by
the \gl{FROM} clause, creating one nested environment with a newly bound
variable for every \gn{var\_decl}. The \emph{last} environment produced by the
\gl{LET} clause is accessible in the remaining clauses of the \gl{SELECT} query
with certain exceptions defined in section
\ref*{sec:let-variable-accessibility}.

\subsubsection{Acessibility of variables defined using \gl{LET}}
\label{sec:let-variable-accessibility}

The \gl{GROUP BY} clause consumes the environments produced by the \gl{FROM} clause and
produces a new set of environments, which affects the clauses which can access
the variables defined by \gl{LET}.  Figure
\ref*{figure:let-variable-accessibility} shows which parts of a \gl{SELECT}
query can access variables defined by \gl{LET} with and without \gl{GROUP
BY}.

\begin{figure}[ht]
\centering
\begin{tabular}{lcc}    
    \gl{SELECT} & \multicolumn{2}{c}{Are variables defined with \gl{LET} accessible?}  \\
    clause & Without \gl{GROUP BY} & With \gl{GROUP BY}  \\
    \hline
    \gl{FROM} & No & No \\
    \gl{LET} & Yes (per lexical scoping) & Yes (per lexical scoping) \\
    \gl{WHERE} & Yes & Yes \\
    \gl{GROUP BY} & N/A & Yes \\ 
    \gl{LETTING} & N/A & No \\
    \gl{HAVING} & N/A & No \\
    \gl{ORDER BY} & Yes & No \\
    \gl{OFFSET} & No & No  \\ 
    \gl{LIMIT} & No & No \\
    projection & Yes & No \\
\end{tabular}
\caption{Accessibility of variables defined using the \gl{LET} clause.}
\label{figure:let-variable-accessibility}
\end{figure}

Note that variables defined by \gl{LET} are not accessible to the \gl{LIMIT} or
\gl{OFFSET} clauses because those clauses do not have access to the environments
produced by the \gl{FROM} or \gl{GROUP BY} clauses.


\subsection{\gl{LETTING} clause}

\gl{LETTING} is an optional clause of a \gl{SELECT} query that immediately
follows the \gl{GROUP BY} clause. \gl{LETTING} executes once for every
environment producted by the \gl{GROUP BY} clause, creating one nested
environment for every \gn{var\_decl}. The \emph{last} environment produced by
the \gl{LETTING} clause is accessible in the remaining clauses of the
\gl{SELECT} query as defined in figure
\ref*{figure:letting-variable-accessibility}.

Use of \gl{LETTING} without \gl{GROUP BY} is considered to be an error.  

\begin{figure}[ht]
\centering
\begin{tabular}{lcc}    
    \gl{SELECT} & Are variables defined with \\
    clause & \gl{LETTING} accessible? \\
    \hline
    \gl{FROM} & No \\
    \gl{LET} & No \\
    \gl{WHERE} & No \\
    \gl{GROUP BY} & No \\ 
    \gl{LETTING} & Yes (per lexical scoping) \\
    \gl{HAVING} & Yes \\
    \gl{ORDER BY} & Yes \\
    \gl{OFFSET} & No  \\ 
    \gl{LIMIT} & No \\
    projection & Yes \\
\end{tabular}
\caption{Accessibility of variables defined using the \gl{LETTING} clause.}
\label{figure:letting-variable-accessibility}
\end{figure}


\subsection{Behavior of \gl{WITH}, \gl{LET} and \gl{LETTING} and \gl{SELECT *}}

\gl{SELECT *} is unaffected by the use of  \gl{WITH}, \gl{LET} and \gl{LETTING}.
That is, variables introduced with these clauses are not included in
\gl{SELECT *} projections.