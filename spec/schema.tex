\section{Structural Types and Type-related Query Syntax and Semantics (WIP)}
\label{sec:schema}
The input data generally conform to a \textit{structural type}, also often
called \textit{schema}. The SQL semantics make extensive use of the structural
types in order to assign meaning to queries, which would not have a meaning in
the absence of such structural types.

\almann{TODO: Add reference to static environment.}

In the interest of SQL compatibility and user convenience, PartiQL also allows
structural types to assign meaning to queries that would not have a meaning
otherwise.

We will soon specify the precise rules that provide SQL compatibility, 
while keeping the schema optional and the query results stable with respect to schema addition.

\eat{
\begin{example}
\label{xmpl:reldb-with-schema}
Consider an example relational database with two tables \gt{sensors} and
\gt{logs}, with respective schemas requiring that each sensors' tuple has an
integer attribute \gt{sensor} and each logs' tuple has an integer \gt{sensor}
attribute and a decimal \gt{co} attribute.%
\footnote{It is the same database used in Example~\ref{xmpl:nesting-readings}.}
\begin{tabbing}
\ \ \ \=\gt{sensors : [}\=\gt{\{'sensor':1\},}\\
\>\>\gt{\{'sensor':2\}}\\
\>\>\gt{]}\\
\ \ \ \=\gt{logs: [}\=\gt{\{'sensor':1, 'co':0.4\},}\\
\>\>\gt{\{'sensor':1, 'co':0.2\},}\\
\>\>\gt{\{'sensor':2, 'co':0.3\}}\\
\>\>\gt{]}
\end{tabbing}
Then the SQL query
\begin{tabbing}
\ \ \ \=\gl{SELECT s.sensor AS sensor, co AS reading}\\
\>\gl{FROM sensors AS s, logs AS l}\\
\>\gl{WHERE s.sensor = l.sensor}
\end{tabbing}
\noindent makes sense semantically, despite the expression \gt{co} not
specifying whether it is \gt{l.co} or \gt{s.co}, since the schema dictates that
only \gt{l} has a \gt{co} attribute. Thus \gt{co} means \gt{l.co}.
\end{example} 

\highlight{Remark} The term ``schema" has different meanings and different roles
in various contexts. In relational databases, the schema declarations are often
overloaded with declarations that dictate the specifics of storage. In querying
S3-based data, the PartiQL schema-as-view proposal overloads
schema-as-constraint to also become a data extraction and cleaning tool. As far
as the semantics of querying schemaful data are concerned, the only aspect of
schema that PartiQL querying cares about, is the constraints that the schema
induces on the input data.

\subsection{Structural Types}
\label{sec:structural-types}
The PartiQL structural type (henceforth called just type) syntax is identical to
the PartiQL literal syntax with the following additions. (We do not provide a
formal BNF syntax and semantics.)

\begin{enumerate}
\item Allow the name of an PartiQL type in lieu of a scalar literal.
\begin{example}
\label{xmpl:optional-date1-attr}
The type 
\begin{tabbing}
\ \ \ \gt{\ob \{'date1': DATE\} \cb} 
\end{tabbing}
describes a collection of tuples that have an optional \gt{date1} attribute and
this attribute has \gt{DATE} values. 
\end{example}

\begin{example}
A type may describe nested data. For example, the following describes a table
where the sales attribute is an array of tuples with an integer \gt{product} and
an integer \gt{amount}.
\begin{tabbing}
\ \ \ \=\gt{\ob \{}\=\gt{date1: DATE,}\\
\>\>\gt{sales:[\{product:INTEGER, amount:DECIMAL\}]}\\
\>\>\gt{\}\cb} 
\end{tabbing}
\end{example}

\item Allow a union type $t_1 | t_2 | \ldots | t_n$ in lieu of a scalar literal.
\begin{example}
\gt{date1} is either a \gt{DATE} or a \gt{INTEGER} according to the following
\begin{tabbing}
\ \ \ \gt{\ob \{date1: DATE | INTEGER\} \cb}
\end{tabbing}
\end{example}

\item Allow a constraint \gl{NOT MISSING} applied on an attribute.
\begin{example}
\label{xmpl:required-date1-attr}
The following type describes a collection of tuples that always have a
\gt{date1} attribute, which may nevertheless have \gl{NULL} value.
\begin{tabbing}
\ \ \ \gt{\ob \{'date1': DATE NOT MISSING\} \cb} 
\end{tabbing}
Contrast with Example~\ref{xmpl:optional-date1-attr}.
\end{example}

\item SQL's \gl{NOT NULL} is applied on any value - this includes attributes
values, as well as bag elements and array elements. Such value cannot be missing
and cannot be null.

\yannis{to almann and chris: This semantics has one downside: It makes it very
difficult to express (and impossible until we introduce named types) to express
that ``attribute $x$ may be missing but if it is there, then it is not null".
The reason I prefer this semantics for \gl{NOT NULL} is the straight
correspondence to SQL's \gl{NOT NULL}.}

\begin{example}
The following type describes a collection of tuples that always have a
\gt{date1} attribute and this attribute is never \gl{NULL}.
\begin{tabbing}
\ \ \ \gt{\ob \{'date1': DATE NOT NULL\} \cb} 
\end{tabbing}
Contrast with Examples~\ref{xmpl:optional-date1-attr}
and~\ref{xmpl:required-date1-attr}.
\end{example}

\item Allow \gl{OPEN} tuple types $\{a_1:t_1,\ldots,a_n:t_n,\gl{OPEN}\}$. The
open tuple type may have more attributes than the attributes $a_1,\ldots,a_n$
that are explicitly listed within its definition.
\begin{example}
The following type is a collection of tuples that absolutely have a \gt{date1}
attribute but they may also have other attributes.
\begin{tabbing}
\ \ \ \gt{\ob \{'date1': DATE NOT NULL, OPEN\} \cb} 
\end{tabbing}
In all prior examples, tuple types were \textit{closed} in the sense that they
explicitly specified the attributes they contain.
\end{example}

\end{enumerate}

\highlight{Future Extensions} Schema and schema-as-view will be expanded in
multiple directions, such as inclusion of \gl{CHECK}, \gl{DEFAULT},
\gl{REFERENCES}, etc. Nevertheless, the present definition of schema captures
any aspect that pertains to PartiQL query semantics.

\subsubsection{SQL's \gl{TABLE} as a special case of tuple collections}
\label{sec:create-table}
SQL's \gl{TABLE} is syntactic sugar for collection of closed tuples. The
following two are equivalent:
\[ \gl{TABLE(}a_1:t_1, \ldots, a_n:t_n\gl{)} \] \noindent and
\[ \ob \{ a_1:t_1, \ldots, a_n:t_n \} \cb \]

The keyword \gl{CREATE} is irrelevant in the context of this schema discussion,
since PartiQL does not necessitate that types specify storage. Eg, the data may
already exist.

\subsection{Tuple Navigations that Omit the Starting Variable}
\label{sec:paths-with-no-starting-variable}
For SQL compatibility, PartiQL allows tuple path navigations that do not
explicitly initiate from a variable; rather, they start from an attribute of the
tuples that bind to the variable. 

\begin{example}
\label{xmpl:sql-attribute-dereferencing}
Consider the SQL table \gt{R} that has an attribute \gt{a} and the table \gt{S}
that does not have an attribute \gt{a}. Then the SQL query
\begin{tabbing}
\ \ \ \=\gl{SELECT a, sv.b}\\
\>\gl{FROM R AS rv, S AS sv}
\end{tabbing}
\noindent is syntactic sugar for
\begin{tabbing}
\ \ \ \=\gl{SELECT rv.a, sv.b}\\
\>\gl{FROM R AS rv, S AS sv}
\end{tabbing}
That is, the path \gt{a} in the \gl{SELECT} clause is dereferenced by SQL into
\gt{rv.a}.
\end{example}

We carry over and generalize this feature for PartiQL. Notice that in PartiQL
not all variables have closed tuple types, as in SQL. This will require a few
extra conditions for the rewriting.

Formally: Consider 
\begin{enumerate}
\item an PartiQL path expression $a s$, where $a$ is an identifier and $s$ is
the (potentially empty) suffix of the path. 

\item this path expression is either (a) not a \gl{FROM} clause expression, or
(b) it is an $\gl{@}as$ expression (most likely in the \gl{FROM} clause).

\item this expression is evaluated in database environment $\db$ and variables
environment $\env$, which defines variables $v_1,\ldots,v_n$
\item a variable $v_{i_a}$ (among $v_1,\ldots,v_n$) has closed tuple type and
the tuple type has a (potentially missing and potentially null) attribute $a$.
Then $as$ will be rewritten into $v_{i_a}.as$ when there is none of the
conflicts described in the next items
\item none of the variables of $\env$ is named $a$ and is at the same scope with
$v_{i_a}$; if there is, then the identifier $a$ of $as$ resolves to the variable
named $a$. That is, the variable takes priority over the dereferenced attribute
when they are both in the same level. (This behavior is unlike Postgres, where a
compilation error would happen.)
\item none of the variables of $\env$ is named $a$ and is within the scope of
$v_{i_a}$, i.e., more local to the expression than the $v_{i_a}$. If it were,
then (as it happens in Postgres), the identifier $a$ of $as$ refers to the
variable $a$.
\item there is no other variable $v_{i'_a}$ that is (a) at the scope of
$v_{i_a}$, (b) is closed tuple-typed and (c) has an attribute named $a$. If
there is, the query cannot compile since we cannot tell whether $a$ should
become $v_{i_a}.a$ or $v_{i'_a}.a$. 
\item there is no other variable $v_{i'_a}$ that is (a) is within the scope of
$v_{i_a}$, (b) is closed tuple-typed and (c) has an attribute named $a$.  That
is, there is no variable that satisfies the conditions of $v_{i_a}$ but is more
local to the expression. (Same with SQL.) 
\item\label{item:stability-req} there is no variable $v'$ within the scope of
$v_{i_a}$ or at the same scope with $v_{i_a}$, such that $v$ can bind to a tuple
that has an attribute $a$. If this condition is not satisfied, we have a
compilation error. 
\end{enumerate}
\noindent If all of the above conditions hold, then $as$ is rewritten into
$v_{i_a}.as$. Notice that dereferencing to $v_{i_a}.as$ also takes priority over
a potential name $a$ in the database environment $\db$. 

\yannis{Item~\ref{item:stability-req} is introduced for ensuring stability upon the imposition of schema.}

\almann{Yannis, I don't understand the above comment.}

Notice that the variable $v'$ of Item~\ref{item:stability-req} has not been
required to have a tuple-type schema with attribut $a$. If the type of variable
$v'$ is any of the following cases, then $v'$ can bind to a tuple that has an
attribute $a$.

\begin{enumerate}
\item type \gl{ANY}
\item open tuple type
\item closed tuple type and this closed tuple type has attribute $a$
\item\label{item:union-cond} union type $t_1 | t_2 | \ldots | t_n$ and, recursively, at least one of the types $t_1, \ldots, t_n$ satisfies a condition from 1-\ref{item:union-cond}
\end{enumerate}

\begin{example} 
Consider the following slight modification of
Example~\ref{xmpl:sql-attribute-dereferencing}: \gt{R} and \gt{S} are still
collections of tuples, but these tuples need not necessarily have scalar values
only. Then the PartiQL query
\begin{tabbing}
\ \ \ \=\gl{SELECT a[5], sv.b}\\
\>\gl{FROM R AS rv, S AS sv}
\end{tabbing}
\noindent is syntactic sugar for
\begin{tabbing}
\ \ \ \=\gl{SELECT rv.a[5], sv.b}\\
\>\gl{FROM R AS rv, S AS sv}
\end{tabbing}
Notice, the path \gt{a[5]} had the identifier \gt{a}, which matched with the
schema of \gt{rv}, and the suffix \gt{[5]}, which is carried over in the
rewriting.
\end{example}

\begin{example}
\label{xmpl:variable-attribute-conflict}
When variable names conflict with identifiers that can be dereferenced to
``variable.attribute" paths, the variable names have priority. This behavior is
unlike Postgres, where the same issue can happen due to allowing variables that
range over arrays of scalars but Postgres would throw a compilation error.

The bag \gt{R} has closed tuples that have an attribute \gt{a}, among others. In
the following query the \gt{a} in the \gt{SELECT} clause stands for the variable
\gt{a} defined in the \gt{FROM} clause.
\begin{tabbing}
\ \ \ \=\gl{SELECT a}\\
\>\gl{FROM R AS rv, @rv.somearray AS a}
\end{tabbing}
\end{example}

\eat{
\begin{example}
The bag \gt{R} has again (as in Example~\ref{xmpl:variable-attribute-conflict})
closed tuples that have an attribute \gt{a}, among others. The following query
compiles, because we cannot tell whether \gt{a} stands for \gt{rv.a} or whether
it stands for the variable \gt{a} in the \gt{FROM} clause.
\begin{tabbing}
\ \ \ \=\gl{SELECT a}\\
\>\gl{FROM R AS rv}\\
\>\gl{WHERE EXISTS(}\=\gl{SELECT *}\\
\>\>\gl{FROM @rv.somearray AS a}\\
\>\>\gl{WHERE a > 5}
\end{tabbing}
The \gt{a} of the \gt{WHERE} clause is the variable \gt{a}, since it is more
local than \gt{rv.a}. The \gt{a} of the \gt{SELECT} clause is rewritten to
\gt{rv.a}.
\end{example}
}

\begin{example}
This example illustrates the stability provided by
Item~\ref{item:stability-req}.

Consider the following SQL query

\begin{tabbing}
\ \ \ \=\gl{SELECT *}\\
\>\gl{FROM gtable AS g}\\
\>\gl{WHERE EXISTS (SELECT * FROM ftable AS f WHERE x = f.y)}
\end{tabbing}

\noindent where \gt{gtable} is a Redshift table with schema \gt{\ob \{x:
INTEGER\} \cb} and \gt{ftable} is an S3 dataset that has no schema, i.e.,
technically its schema is \gl{ANY}. The \gt{x} in the \gl{WHERE} clause will not
be rewritten into \gt{g.x} and this query fail compilation due to the
unresolveable reference \gt{x}. Technically, the reason that it will not be
rewritten to \gt{g.x} is because there is the possibility that \gt{f} can also
bind to tuples that have \gt{x} (Item~~\ref{item:stability-req}).

Intuitively, the reason for not rewriting is the instability that would have
happened, had we allowed the rewriting. Then \gt{x} would be rewritten into
\gt{g.x}. Then the query would be equivalent to 

\begin{tabbing}
\ \ \ \=\gl{SELECT *}\\
\>\gl{FROM gtable AS g}\\
\>\gl{WHERE EXISTS (SELECT * FROM ftable AS f WHERE g.x = f.y)}
\end{tabbing}

Now, someone (different guy from the one who wrote the query) inspects the
\gt{ftable} and declares the schema \gt{\ob \{x: INTEGER\} \cb} for \gt{ftable}.
Now \gt{x} would resolve to \gt{f.x} and the query would be rewritten into

\begin{tabbing}
\ \ \ \=\gl{SELECT *}\\
\>\gl{FROM gtable AS g}\\
\>\gl{WHERE EXISTS (SELECT * FROM ftable AS f WHERE f.x = f.y)}
\end{tabbing}

Therefore, such query semantics are unstable in the sense that they change by
the mere definition of a schema. 
\end{example}

\highlight{Remark} The collective path expression rewriting
(Section~\ref{sec:deep-navigation}) preceeds the schema-based rewriting,
described in the present section.

\subsection{Attribute Ordinals in lieu of Attribute Names}
\label{sec:ordinals-in-lieu-of-names}
In certain SQL operations, an expression may refer to an attribute via its
ordinal position, as defined by the schema (\gl{CREATE TABLE}). PartiQL achieves
the same by utilizing schemas in order to statically map ordinal positions into
names. PartiQL does not allow the mapping in the absence of schemas, in order to
avoid unstable and expensive-to-execute semantics.

Consider a collection of tuples, where each tuple follows the closed tuple
schema $\{a_1:t_1,\ldots,a_n:t_m\}$. Such a collection is called a \textit{table
collection}. The following ``syntactic sugar" semantics apply to table
collections.

\subsubsection{Set Operations on Table Collections}
\label{sec:setops-on-tables}
Consider any set operation $\circ$ (\gl{UNION}, \gl{UNION ALL}, \gl{CONCAT},
etc) applied on two table collections $c^1$ and $c^2$ that \textit{both} have
schema. Furthermore, assume the schema of the tuples of $c^1$ is
$\{a_1^1:t_1^1,\ldots,a_n^1:t_n^1\}$ and the schema of the tuples of $c^2$ is
$\{a_1^2:t_1^2,\ldots,a_n^2:t_n^2\}$. Then the user should think of the
schemaful tuples as fixed-arity arrays, where the attribute names define the
order of the values in the array. In particular, $c^1 \circ c^2$ is equivalent
to

\begin{tabbing}
\ \ \ \=\gl{SELECT VALUE \{}$x[0]$ \gl{AS} $a_1^1\gl{,}\ldots\gl{,} x[n-1]$
\gl{AS} $a_n^1$\gl{\}}\\
\>\gl{FROM (}\=\gl{SELECT VALUE [}$v.a_1^1\gl{,} \ldots\gl{,} v.a_n^1$\gl{] FROM
}$c^1$ \gl{AS }$v$\\
\>\>$\circ$\\
\>\>\gl{SELECT VALUE [}$w.a_1^2\gl{,} \ldots\gl{,} w.a_n^2$\gl{] FROM }$c^2$
\gl{AS }$w$\\
\>\>\gl{) AS }$x$\\
\end{tabbing}
 
We can also rewrite $c^1 \circ c^2$ as
\[ c^1 \circ (\gl{SELECT}\ v.a_1^2\ \gl{AS}\ a_1^1, \ldots, v.a_n^2\ \gl{AS}\
a_n^1\ \gl{FROM}\ c^2 \gl{ AS } v) \] \noindent i.e., the set operation that
results from renaming the attribute names of the second argument into the
attribute names of the first argument. The two rewritings are equivalent.

Note that the collection schemas may not necessarily be declared; rather, they
may be inferred, as illustrated in the following example.

\begin{example}
Consider the query
\begin{tabbing}
\ \ \ \=\gl{SELECT v.a AS a, v.b AS b FROM c1 v}\\
\>\gl{UNION}\\
\>\gl{SELECT w.b AS b, w.a AS a FROM c2 w}
\end{tabbing} 
Assume there is no schema for \gl{c1} and \gl{c2}. Still, the first argument of
the \gl{UNION} has schema \gl{\ob \{a:ANY, b:ANY\}\cb} and the second one has
schema \gl{\ob \{b:ANY, a:ANY\}\cb}. Thus they are both collection tables.
Consequently, the above query is rewritten into
\begin{tabbing}
\ \ \ \=\gl{SELECT v.a AS a, v.b AS b FROM c1 v}\\
\>\gl{UNION}\\
\>\gl{SELECT w.b AS a, w.a AS b FROM c2 w}
\end{tabbing} 
\end{example}

\yannis{I believe this stability issue that emerges from the combination of  is
unavoidable.} Notice that SQL (and, for compatibility purposes, PartiQL) uses
the ordinals, even if the two collections' tuples have the same attributes. The
alteration of semantics by the presence of schemas can cause stability issues in
the sense that the introduction of a schema may alter the semantics of a query
that does not explicitly enumerate the attributes in its result, as illustrated
by the following example: 

\begin{example}
Consider two collections \gl{c1} and \gl{c2} with no schema and the following
data:
\begin{tabbing}
\ \ \ \=\gl{c1: \ob \{a:1, b:2\} \cb}\\
\ \ \ \=\gl{c1: \ob \{a:1, b:2\} \cb}
\end{tabbing}
The following query uses \gl{*} and thus does not explicitly enumerate the
attributes in the output
\begin{tabbing}
\ \ \ \=\gl{SELECT * FROM c1 v}\\
\>\gl{UNION}\\
\>\gl{SELECT * FROM c2 w}
\end{tabbing} 
\noindent It results into \gl{\{a:1, b:2\}}.

\yannis{Explain step by step that the * was expanded in the presence of schema}
Then, we give the following schemas to the two collections:
\begin{tabbing}
\ \ \ \=\gl{c1: \ob \{a:ANY, b:ANY\} \cb}\\
\ \ \ \=\gl{c2: \ob \{b:ANY, a:ANY\} \cb}
\end{tabbing}
Notice that the tuples are now ordered. Thus it matters that in \gl{c1} the
attribute \gl{a} is first and the attribute \gl{b} is second, while in \gl{c2}
the attribute order is vice versa. The result of the same \gl{UNION} query is
\begin{tabbing}
\ \ \ \=\gl{\ob }\=\gl{\{a:1, b:2\}}\\
\ \ \ \>\>\gl{\{a:1, b:2\} \cb}
\end{tabbing}
\noindent because it has the schema of the left argument of the \gl{UNION} 
\end{example}

}
















