\section{Scoping rules}
\label{sec:variable-scoping}

As far as the variables environment is concerned, the scoping rules are
identical to those of SQL. Section~\ref{sec:scoping-variables} explained how the
resolution of variable naming conflicts favors the variables defined by the
inner queries.

The scoping rules discussed in the present section discuss the resolution of
naming conflicts between names defined in the database environment and the
variables of the environment variables. The potential for such naming conflicts
is driven by the nested data of PartiQL, as illustrated next.

Notice there are a few more naming conventions, pertaining to the use of
attribute names defined in the \gl{SELECT} clause into the \gl{GROUP BY} and
\gl{ORDER BY} clause. These conventions are explained in along with the
semantics of the respective clauses (see Sections~\ref{section:group-by}
and~\ref{section:order-by}).

\almann{TODO: Clean up this terminology with respect to
Section~\ref{sec:environments-and-bindings}}

\begin{example}
The following example illustrates how SQL compatibility issues and the needs of
navigating into nested data need to be carefully merged together. Consider the
following database that has a table \gt{c}, i.e. a collection of tuples, and
also named data \gt{x.n} and \gt{y}.

\begin{tabbing}
\ \ \ \=\gt{t.c: \ob\ }\=\gt{\{'a':1, 'n':[\{'b':11, 'c':12\}]\},}\\
\>\>\gt{\{'a':2, 'n':[\{'b':21, 'c':22\}]\}}\\
\>\>\gt{\cb}\\
\>\gt{x.n : \ob\ \{'b':3\}\ \cb}\\
\>\gt{y: \{'a':1, 'b':2\}}
\end{tabbing}

Then consider the query
\begin{tabbing}
\ \ \ \=\gl{SELECT t.a}\\
\>\gl{FROM t.c AS x}\\
\>\gl{WHERE x.a IN (SELECT y.b FROM x.n AS y)}
\end{tabbing}

This query poses many scoping issues:
\begin{enumerate}
\item Does \gt{x.n} refer to the named value \gt{x.n} or to the \gt{n} attribute
of the variable \gt{x}? For SQL compatibility purposes it refers to the named
value \gt{x.n}. Read below how to refer to the variable \gt{x}.
\item Does \gt{y.b} refer to the \gt{b} attribute of the \gt{y} attribute or to
the \gt{b} attribute of the named value \gt{y}? For SQL compatibility purposes
it refers to the \gt{b} attribute of the variable \gt{y}.
\end{enumerate}
Notice how SQL compatibility required the database environment to take priority
over the variables environment in the \gl{FROM} clause and then, vice versa, the
variables environment to take priority over the database environment in the
\gl{SELECT} clause.
\end{example}

\highlight{Scoping rules resolving naming conflicts between variables and database names}
Since the rules are easier to express when all database names are a single
identifier, such as \gt{thedb} or \gt{"the db"} (as opposed to paths, such as
\gt{somedb.sometable}), we first specify the scoping rules under the assumption
that all database names are a single identifier. We remove the assumption and
generalize later.

In the absence of schema the following rules apply
\begin{enumerate}
\item \gl{@}\gn{identifier} refers to the environment variable named
\gn{identifier}; if there is no such environment variable, the \gn{identifier}
refers to the database name \gn{identifier}; if there is no such database name
either, the query fails compilation.
\item in a \gl{FROM} clause path that starts with \gn{identifier}, the
\gn{identifier} refers to the database name \gn{identifier}; if there is no such
database name, the \gn{identifier} refers to a variable; otherwise query fails
compilation. \footnote{A path is a \from\ clause path if it appears in the \from
\ clause of the SFW query in which it is \textit{immediately} nested.}
\item in a non-\from\ clause path that starts with \gn{identifier}, the
\gn{identifier} refers to the environment variable named \gn{identifier}; if
there is no such environment variable, the \gn{identifier} refers to the
database name \gn{identifier}; if there is no such database name either, the
query fails compilation.
\end{enumerate}

Next, we
generalize to also allow for the possibility of database names of the form
$\gn{identifier}\gl{.}\gn{indentifier}\gl{.}\ldots$. The following rules apply
regarding the semantics of $i_1 \gl{.} i_2 \gl{.} \ldots \gl{.} i_n$, where
$i_1, i_2, \ldots i_n$ are identifiers.
\begin{itemize}
\item \gl{@}$i_1 \gl{.} i_2 \gl{.} \ldots \gl{.} i_n$ always refers to the
environment variable named $i_1$; if there is no such variable and $i_1 \gl{.}
i_2 \gl{.} \ldots \gl{.} i_m, m\leq n$ is a database name then $i_1 \gl{.} i_2
\gl{.} \ldots \gl{.} i_m$ refers to such named database name. Again, if there is
a choice, choose the largest $m$. If both the resolution to variable and the
resolution to database name, fail the query during compilation.
%
\item if  $i_1 \gl{.} i_2 \gl{.} \ldots \gl{.} i_n$ is a \from\ path and $i_1
\gl{.} i_2 \gl{.} \ldots \gl{.} i_m, m\leq n$ is a database name then $i_1
\gl{.} i_2 \gl{.} \ldots \gl{.} i_m$ refers to such named database name and
$i_{m+1} \gl{.} \ldots \gl{.} i_n$ is a series of tuple path navigations
starting from the database name $i_1 \gl{.} i_2 \gl{.} \ldots \gl{.} i_m$. If
there is a choice, choose the largest $m$, i.e., the longest database name.
%
\item if $i_1 \gl{.} i_2 \gl{.} \ldots \gl{.} i_n$ is a non-\from\ clause
expression and $i_1$ is an environment variable then $i_1$ refers to such
variable; if there is no such variable and $i_1 \gl{.} i_2 \gl{.} \ldots \gl{.}
i_m, m\leq n$ is a database name then $i_1 \gl{.} i_2 \gl{.} \ldots \gl{.} i_m$
refers to such named database name. Again, if there is a choice, choose the
largest $m$. If both the resolution to variable and the resolution to database
name, fail the query during compilation.
\end{itemize}

\begin{example}
Assume database names \gt{coll}, \gt{v.foo}, \gt {w}. Then in the query

\begin{verbatim}
1 SELECT v.foo
2 FROM coll AS v, @v.foo AS w, 
3          (SELECT w.a, u.b FROM @w.bar AS u)
4          AS x
\end{verbatim}

\noindent  \gt{coll} refers to the database name. The \gt{v} in \gt{@v.foo} refers to the variable \gt{v}. If the \gt{@} were not there, \gt{v.foo} would refer to the database name \gt{v.foo}. The \gt{w} in \gt{w.a} refers to the variable defined in line~2. 

Note, the expressions \gt{coll} and \gt{@v.foo} are \from\ clause expressions because they appear in the \from\ clause of the \gn{sfw\_query} of lines~1-4, in which they are immediately nested. Similarly, the expression \gt{@w.bar} is a \from\ clause expression because it appears in the \from\ clause of the \gn{sfw\_query} of line~3, in which it is immediately nested. In contrast, the expressions \gt{w.a} and \gt{u.b} are not \from\ clause expressions. Though they are nested into the \from\ clause of the query of lines~1-4, they are not immediately nested into the query of lines~1-4. 
\end{example}
