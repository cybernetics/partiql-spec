\section{\gl{GROUP BY} clause}
\label{section:group-by} 
The PartiQL \gl{GROUP BY} clause expands SQL's grouping. Unlike SQL, the PartiQL
\gl{GROUP BY} can be thought of as a standalone operator that inputs a
collection of binding tuples and outputs a collection of binding tuples. 

As is typical in many clauses, the semantics proceed in two steps:
\begin{itemize}
\item Section~\ref{sec:group-variable} explains the core PartiQL \gl{GROUP BY}
structure.
\item Section~\ref{sec:sql-groupby} shows that SQL's \gl{GROUP BY} can be
explained over the core \gl{GROUP BY}.
\end{itemize}

\subsection{PartiQL \gl{GROUP BY} core: Grouping into a Group Variable}
The \gl{GROUP BY} clause \linequery{group by} 
\label{sec:group-variable}

%
\[ \gl{GROUP BY}\ e_1\ \gl{AS}\ x_1\ \gl{,} \ldots \gl{,}\ e_m\ \gl{AS}\ x_m\ \gl{GROUP AS}\ g\] 
%
\noindent creates a group. Each $e_i$ is a \textit{grouping expression}, each
$x_i$ is a \textit{grouping variable}
\footnote{Grouping variables is an extension of SQL by PartiQL, which
interestingly simplifies dramatically the explanation of SQL semantics, as it
enables the \gl{GROUP BY} to be seen as a standalone function.}
and $g$ is the \textit{group variable}. 

\almann{TODO define the semantics when the group expression is a constant
positive integer--specifically the relation to the \gl{SELECT} projection
items.
Yannis: Interestingly, this is a deprecated feature of the SQL standard as of 1998.
Unfortunately, Postgres carried the torch but I think that all we have to do 
is to say ``avoid using positive constants" as many implementations 
use them to mean the first column of SELECT. 
Recall, this was Don's March feedback, that
we cannot hold Couchbase for SQL incompatibility on what is not SQL :) }

As in SQL, the bag of input binding tuples $B^{in}_{\gl{GROUP}}$ is partitioned
into the minimal number of equivalence groups $B_1 \ldots B_n$, such that any
two binding tuples $b, b' \in B^{in}_{\gl{GROUP}}$ are in the same equivalence
group if and only if every grouping expression $e_i$ evaluates to equivalent
values $v_i$ (when evaluated on $b$) and $v'_i$ (when evaluated on $b'$). More
precisely, as in SQL, there is an equivalence function \gl{eqg}, used by the
\gl{GROUP BY} to determine if two values $v_i$ and $v_i'$ are equivalent for
grouping purposes. The equivalence function $\gl{eqg(}v_i, v'_i\gl{)}$ returns
only true or false; true meaning that the values are equivalent for grouping
purposes. See Section~\ref{sec:eqg} for specifics of \gl{eqg}. If a grouping
expression evaluates to \MISSING, it is first coerced into \NULL,
thus bringing \MISSING and \NULL in the same group.

Unlike SQL, for each group $B_j ~ (1 \leq j \leq n)$, the \gl{GROUP BY} clause
outputs a binding tuple $b_j = \langle x_1 : v_1, ~ \ldots, ~ x_m : v_m, ~ g :
B_j \rangle$ that has the full group $B_j$. Notice:
\begin{enumerate}
\item the binding tuples that appear in the $g$ collection have one attribute
for each of the variables defined in the \gl{FROM} clause,
since these binding tuples come as-is from $B^{in}_{\gl{GROUP}}$.
\item even if the bag $B^{in}_{\gl{GROUP}}$ is flat binding tuples, the output
bag $B^{out}_{\gl{GROUP}}$ is not just flat binding tuples, since $g$ has nested
binding tuples. Note, we have been explicitly denoting binding attributes
with \MISSING values in the binding tuples. However, once these binding tuples
become the tuples of the PartiQL data model, any binding attribute with 
\MISSING value will not appear.
\end{enumerate}

\begin{example}
\label{sec:grouping-readings}
Consider again the \gt{logs} data of Example~\ref{xmpl:nesting-readings} and
assume that we want to group the \gt{co} readings by sensor. The following query
solves the problem using only core features. 

\begin{tabbing}
\ \ \ \=\gl{SELECT VALUE \{}\=\gl{'sensor': sensor,}\\
\>\>\gl{'readings': (SELECT VALUE v.l.co FROM g AS v) \}}\\
\>\gl{FROM logs AS l}\\
\>\gl{GROUP BY l.sensor AS sensor GROUP AS g}
\end{tabbing}
The \gl{GROUP BY} outputs the collection of binding tuples
\begin{tabbing}
\ \ \ \=$B^{out}_{\gl{GROUP}} = B^{in}_{\gl{SELECT}} =$\=\gt{\ob }\=\gt{$\langle$ sensor:1, g: \ob}\=\gt{$\langle$l:\{'sensor':1, 'co':0.4\}$\rangle$,}\\ 
\>\>\>\>\gt{$\langle$l:\{'sensor':1, 'co':0.2\}$\rangle$}\\
\>\>\>\>\gt{\cb $\rangle$,}\\
\>\>\>\gt{$\langle$ sensor:2, g:\ob $\langle$l:\{'sensor':2, 'co':0.3\}$\rangle$ \cb $\rangle$}\\
\>\>\gt{\cb}
\end{tabbing}

Notice that the collection \gt{g} has tuples with a single attribute \gt{l},
since this is the single variable of the \gl{FROM} clause in this example.

Consequently the \gl{SELECT} clause outputs
\begin{tabbing}
\ \ \ \=\gt{\ob }\=\gt{\{'sensor':1, 'readings':\ob 0.4, 0.2\cb\},}\\
\>\>\gt{\{'sensor':2, 'readings':\ob 0.3 \cb\}}\\
\>\gt{\cb}
\end{tabbing}
Notice that the query of Example~\ref{xmpl:nesting-readings} and the query of
the present example do not always produce the same result. For example, if there
were no readings for a sensor, the query of Example~\ref{xmpl:nesting-readings}
would still have this sensor in the result (and its \gt{readings} would be
empty). In contrast, the query of the present example will not have this sensor
in the result.

Here is a shorter equivalent query that uses PartiQL collection paths and SQL's
aliases.
\begin{tabbing}
\ \ \ \=\gl{SELECT VALUE \{}\=\gl{'sensor': sensor,}\\
\>\>\gl{'readings': g[*].l.co \}}\\
\>\gl{FROM logs AS l}\\
\>\gl{GROUP BY l.sensor AS sensor GROUP AS g}
\end{tabbing}
\end{example}

Notice, the output binding tuple provides the partitioned input binding tuples
in the group variable $g$, which can be explicitly utilized in subsequent
\gl{HAVING}, \gl{ORDER BY} and \gl{SELECT} clauses. Thus, an PartiQL query can
perform complex computations on the groups, leading to results of any type (e.g.
collections nested within collections). The explicit presence of groups in
PartiQL, while more general than SQL, also leads to simpler semantics than those
of SQL, since the \gl{GROUP BY} clause semantics are independent of the presence
of subsequent functions in \gl{HAVING}, \gl{ORDER BY} and \gl{SELECT}.

\begin{example}
\label{xmpl:groupby-avg-count}

The following PartiQL query counts and averages the readings of each sensor. It
also refers to the \gt{logs} of Example~\ref{xmpl:nesting-readings}. The
\gl{COLL\_COUNT} function is simply given the group variable and counts how many
elements are in that collection.
\begin{tabbing}
\ \ \ \=\gl{SELECT VALUE \{}\=\gl{'sensor': sensor,}\\
\>\>\gl{'avg': COLL\_AVG(SELECT VALUE v.l.co FROM g AS v), }\\
\>\>\gl{'count': COLL\_COUNT(g) \}}\\
\>\gl{FROM logs AS l}\\
\>\gl{GROUP BY l.sensor AS sensor GROUP AS g}
\end{tabbing}
\end{example}

Notice, the aggregate functions \gl{COLL\_AVG} and \gl{COLL\_COUNT} (and for
that matter, by convention, any function starting with \gl{COLL}) can be thought
of as general-purpose functions. Generally, they do not have to be fed by the
result of a grouping operation - unlike SQL's \gl{COUNT} and \gl{AVG} that are
being fed exclusively from the results of grouping operations. (Furthermore, the
SQL \gl{COUNT} and \gl{AVG} make use of SQL's syntactic sugar, where there is no
explicit use of group variable, as explained in
Section~\ref{sec:implicit-group-variable}.)

\begin{example} 
This is a legitimate PartiQL expression:
\begin{tabbing} 
\ \ \ \gl{COLL\_COUNT(}\gt{[5, \{a:2, b:3\}]}\gl{)}
\end{tabbing}
The result is \gt{2}, since the input to \gl{COLL\_COUNT} is an array with two
elements. 
\end{example}
Similarly, it is fine to include in any clause an aggregate function fed by the
result of a (sub)query.
\begin{example}
In the following expression \gl{COLL\_COUNT} inputs the result of a query
\begin{tabbing}
\ \ \ \gl{COLL\_COUNT(SELECT VALUE x FROM logs x WHERE x.sensor=1)}
\end{tabbing}
\end{example}

\highlight{Remark} An efficient implementation will often avoid materializing
the group variable. In many cases, like the ones of the above examples, the
group can be streamed into the aggregate function.

\highlight{Remark} The semantics of the \gl{SELECT} and \gl{HAVING} clauses do
not need to be aware of the presence of \gl{GROUP BY} and treat differently (as
SQL would do) these classes of functions: 
\begin{itemize}
\item scalar functions (e.g. \gl{+}) that input scalars and output scalars 
\item SQL aggregation functions (e.g. \gl{SUM}) that input bags and output
scalars
\end{itemize}
Indeed, \gl{HAVING} behaves identical to a \gl{WHERE}, once groups
are already formulated earlier.

The PartiQL approach provides two benefits: First, it leads to shorter, modular
semantics. Second, it enables \gl{GROUPY} to address use cases that would
otherwise need knowledge and non-trivial SQL programming of window functions.
See Example~\ref{xmpl:windows-by-grouping}.

\subsubsection{Equivalence function used by grouping; grouping of \NULL and \MISSING}
\label{sec:eqg}

The equivalence function \gl{eqg} extends SQL's respective function. In
particular, it behaves as follows:
\begin{itemize}
\item $\gl{eqg}(\NULL, \NULL)$ is true, despite $\NULL=\NULL$ not being true.
\item for any two non-null values $x$ and $y$, $\gl{eqg}(x,y)$ returns the same
with $x=y$. As is the case generally for $=$, while SQL's $=$ will error when
given incompatible types, while the PartiQL $=$ will return \gl{false}.
\end{itemize}

Notice that PartiQL will group together the \NULL and the \MISSING grouping
expressions, since any grouping expression resulting to \MISSING has been
coerced into \NULL before \gl{eqg} does comparisons for grouping.
Example~\ref{xmpl:grouping-null-missing} shows the repercussions of coercing
\NULL into \MISSING and also shows how to discriminate between \NULL and
\MISSING, if so desired.

\begin{example}
\label{xmpl:grouping-null-missing}
The query of Example~\ref{sec:grouping-readings} will group together any log
readings where the \gl{sensor} attribute is either \NULL or is altogether
\MISSING. For example, if \gl{logs} is
\begin{tabbing}
\gl{logs:}\=\gl{[ }\=\gl{\{'sensor': 1, 'co':0.4\},}\\
\>\>\gl{\{'sensor': 2, 'co':0.3\},}\\
\>\>\gl{\{'sensor': null, 'co':0.1\},}\\
\>\>\gl{\{'sensor': 1, 'co':0.2\},}\\
\>\>\gl{\{'co':0.5\}}\\
\>\gl{]}
\end{tabbing}
\noindent then the \gl{GROUP BY} will output the collection of binding tuples
\begin{tabbing}
\ \ \ \=$B^{out}_{\gl{GROUP}} = B^{in}_{\gl{SELECT}} =$\=\gt{\ob }\=\gt{$\langle$ sensor:1, g: \ob}\=\gt{$\langle$l:\{'sensor':1, 'co':0.4\}$\rangle$,}\\ 
\>\>\>\>\gt{$\langle$l:\{'sensor':1, 'co':0.2\}$\rangle$}\\
\>\>\>\>\gt{\cb $\rangle$,}\\
\>\>\>\gt{$\langle$ sensor:2, g:\ob $\langle$l:\{'sensor':2, 'co':0.3\}$\rangle$ \cb $\rangle$,}\\
\>\>\>\gt{$\langle$ sensor:null, g: \ob}\=\gt{$\langle$l:\{'sensor':null, 'co':0.1\}$\rangle$,}\\
\>\>\>\>\gt{$\langle$l:\{'co':0.5\}$\rangle$}\\
\>\>\>\>\gt{\cb $\rangle$} \\
\>\>\gt{\cb}\\
\end{tabbing}
\noindent Notice that both the 3rd and 5th tuples of \gl{logs} were grouped
under the \gl{sensor:null} group, despite the \gl{sensor} of the 3rd being \NULL
while the \gl{sensor} of the 5th being \MISSING. The query result is 
\begin{tabbing}
\ \ \ \=\gt{\ob }\=\gt{\{'sensor':1, 'readings':\ob 0.4, 0.2\cb\},}\\
\>\>\gt{\{'sensor':2, 'readings':\ob 0.3\cb\},}\\
\>\>\gt{\{'sensor':null, 'readings':\ob 0.1, 0.5\cb \}}\\
\>\gt{\cb}
\end{tabbing}

If we wanted to discriminate the \NULL from the \MISSING we could write the
following query
\begin{tabbing}
\ \ \ \=\gl{SELECT VALUE \{}\=\gl{'sensor': CASE WHEN missingFlag THEN MISSING ELSE sensor END,}\\
\>\>\gl{'readings': (SELECT VALUE v.l.co FROM g AS v) \}}\\
\>\gl{FROM logs AS l}\\
\>\gl{GROUP BY l.sensor IS MISSING AS missingFlag, l.sensor AS sensor GROUP AS g}
\end{tabbing}
\noindent In this case the \gl{GROUP BY} would output the collection of binding tuples
\begin{tabbing}
\ \ \ \=$B^{out}_{\gl{GROUP}} = B^{in}_{\gl{SELECT}} =$\=\gt{\ob }\=\gt{$\langle$}\=\gt{missingFlag:false,}\\
\>\>\>\>\gt{sensor:1, g: \ob}\=\gt{$\langle$l:\{'sensor':1, 'co':0.4\}$\rangle$,}\\ 
\>\>\>\>\>\gt{$\langle$l:\{'sensor':1, 'co':0.2\}$\rangle$}\\
\>\>\>\>\gt{\cb $\rangle$,}\\
\>\>\>\gt{$\langle$ missingFlag: false, sensor:2, g:\ob $\langle$l:\{'sensor':2, 'co':0.3\}$\rangle$ \cb $\rangle$}\\
\>\>\>\gt{$\langle$ missingFlag: false, sensor:null, g:\ob $\langle$l:\{'sensor':null, 'co':0.1\}$\rangle$ \cb $\rangle$}\\
\>\>\>\gt{$\langle$ missingFlag: true, sensor:null, g:\ob $\langle$l:\{'co':0.5\}$\rangle$ \cb $\rangle$}\\
\>\>\gt{\cb}\\
\end{tabbing}
\noindent and the query result would be
\begin{tabbing}
\ \ \ \=\gt{\ob }\=\gt{\{'sensor':1, 'readings':\ob 0.4, 0.2\cb\},}\\
\>\>\gt{\{'sensor':2, 'readings':\ob 0.3\cb\},}\\
\>\>\gt{\{'sensor':null, 'readings':\ob 0.1\cb\},}\\
\>\>\gt{\{'readings':\ob 0.5\cb\}}\\
\>\gt{\cb}
\end{tabbing}
\end{example}

\subsubsection{The \gl{GROUP ALL} variant}
\label{sec:group-all}

The \gl{GROUP ALL} variant of \gl{GROUP BY} outputs a single binding tuple,
regardless of whether the \gl{FROM}/\gl{WHERE} produced any tuples, i.e.,
regardless of whether its input $B^{in}_{\gl{GROUP}}$ is empty or not.

The \gl{GROUP ALL} is not increasing the expressiveness of PartiQL.
Example~\ref{xmpl:group-all-core} shows how to achieve without \gl{GROUP ALL},
what the \gl{GROUP ALL} can do. However, we include \gl{GROUP ALL} for
facilitating the reduction of SQL's aggregation into the core PartiQL (see
Section~\ref{sec:implicit-group-variable}). 
%
\begin{example}
\label{xmpl:group-all-core}
Consider again the \gt{logs} data of Example~\ref{xmpl:nesting-readings} and
assume that we want to count the total number of readings that are above
\gt{1.5} with a core PartiQL query. (Example~\ref{xmpl:group-by-nothing-sql}
does the same with SQL.) 
\begin{tabbing}
\ \ \ \=\gl{SELECT VALUE \{'largeco': COLL\_COUNT(g)\}}\\
\>\gl{FROM logs AS l}\\
\>\gl{WHERE l.co > 1.5}\\
\>\gl{GROUP ALL AS g} 
\end{tabbing}
Notice, there are no readings above \gt{1.5} in the example data. Since there is
no tuple that satisfies the \gl{WHERE} clause
\[\begin{array}{l}
B^{out}_{\gl{WHERE}} = B^{in}_{\gl{GROUP}} = \ob\ \cb \\
B^{out}_{\gl{GROUP}} = B^{in}_{\gl{SELECT}} = \ob \langle \gt{g}: \ob\ \cb \rangle \cb
\end{array}
\]
Since $\gl{COLL\_COUNT(}\ob\ \cb\gl{)}$ is \gt{0}, the query result is the collection 
\[\ob \gt{\{'largeco': 0 \}} \cb \]
Therefore the PartiQL query is equivalent to the plain SQL query
\begin{tabbing}
\ \ \ \=\gl{SELECT COUNT(*) AS largeco}\\
\>\gl{FROM logs AS l}\\
\>\gl{WHERE l.co > 1.5}\\
\end{tabbing}
The following core PartiQL also accomplishes the same computation, without using
\gl{GROUP ALL}.
\begin{tabbing}
\ \ \ \=\gl{\{'largeco': COLL\_COUNT(}\=\gl{SELECT VALUE l}\\
\>\>\gl{FROM logs AS l}\\
\>\>\gl{WHERE l.co > 1.5}\\
\>\>\gl{)}\\
\>\gl{\}}
\end{tabbing}
\end{example}

 
\subsection{SQL compatibility features} 
\label{sec:sql-groupby}
The group-by and aggregation of PartiQL is backwards compatible to SQL.

\subsubsection{Grouping Attributes and Direct Use of Grouping Expressions}
\label{sec:grouping-attributes}
For SQL compatibility PartiQL allows

\[\gl{GROUP BY}\ \ldots\gl{,} e\gl{,} \ldots\]

\noindent i.e., a grouping expression $e$ that is not associated with a grouping
variable $x$. (In core PartiQL, one would write $e\ \gl{AS}\ x$.)

For SQL compatibility, PartiQL supports using the grouping expression $e$ in
\gl{HAVING}, \gl{ORDER BY} and \gl{SELECT} clauses. The SQL form (S) is
syntactic sugar for the core PartiQL (C).

\begin{tabular}{@{}l@{~}l@{~}l@{~}l@{}}
(S) & \texttt{FROM ~~~} \ldots                          & (C)   & \texttt{FROM ~~~} \ldots \\
    & \texttt{GROUP BY} $e \texttt{,} \ldots$         &       & \texttt{GROUP BY} $e \texttt{ AS } x \texttt{,} \ldots$ \\
    & \texttt{HAVING ~} $f(e, \ldots)$                &       & \texttt{HAVING ~} $f(x, \ldots)$ \\
    & \texttt{ORDER BY} $f'(e) \texttt{,} \ldots$     &       & \texttt{ORDER BY} $f'(x) \texttt{,} \ldots$  \\
    & \texttt{SELECT ~} $f''(e) \texttt{,} \ldots$    &       & \texttt{SELECT ~} $f''(x) \texttt{,} \ldots$ \\
\end{tabular}

\begin{example}
The SQL-compatible query 
\begin{tabbing}
\ \ \ \=\gl{SELECT v.a+1 AS bar}\\
\>\gl{FROM foo AS v}\\
\>\gl{GROUP BY v.a+1}
\end{tabbing}
\noindent is written in core PartiQL as
\begin{tabbing}
\ \ \ \=\gl{SELECT VALUE \{'bar': x\}}\\
\>\gl{FROM foo AS v}\\
\>\gl{GROUP BY v.a+1 AS x GROUP AS dontcare}
\end{tabbing}
\end{example}

\highlight{Remark: What is ``same expression"?} An open question in the
equivalence of (C) and (S) is the exact meaning of ``same expression $e$ in
\gl{GROUP BY} and \gl{SELECT} (or \gl{HAVING}, \gl{ORDER BY})". Is \gt{v.a + 1}
the same with \gt{1 + v.a}? Is \gt{v.a + 1} the same with \gt{a + 1} in the
presence of a schema that dictates that the variable \gt{v} is a tuple with an
attribute \gt{a}? Both SQL and PartiQL answer ``no" and ``yes" respectively to
the two questions. In particular:

An expression $e$ that appears in the \gl{GROUP BY} clause and an expression
$e'$ that appears in the \gl{SELECT} or \gl{HAVING} or \gl{ORDER BY} are
considered the same expression if they are syntactically identical after
performing the schema-based rewritings of Section~\ref{sec:schema}.

\subsubsection{SQL's Implicit Use of the Group Variable in SQL Aggregate Functions} 
\label{sec:implicit-group-variable}

SQL does not have explicit group variables. For SQL compatibility, PartiQL
allows the SQL aggregation functions to be fed by expressions that do not
explicitly say that there is iteration over the group variable. Suppose that a
query

\begin{enumerate}
\item is a \gl{SELECT} query,
\item lacks a \gl{GROUP AS} clause, and
\item any of the \gl{SELECT}, \gl{HAVING} and/or \gl{ORDER BY} clauses contains
a function call $f(e)$, where $f$ is a \textit{SQL aggregation function} such as
\gl{SUM} and \gl{AVG}. (See Section~\ref{sec:SQL-aggregation-functions} 
\end{enumerate}

Then, the query is rewritten as follows:
\begin{itemize}
\item if the query has a \gl{GROUP BY} clause, add to it 
\begin{tabbing}
\ \ \ \gl{GROUP AS }$g$
\end{tabbing}

\noindent where $g$ is a fresh variable, i.e., a variable that is not a database
name nor a variable of the query or a variable of the queries within which it is
nested.

\item if the query has no \gl{GROUP BY} clause, add to it 
\begin{tabbing}
\ \ \ \gl{GROUP ALL GROUP AS }$g$
\end{tabbing}
\noindent where $g$ is a fresh variable.

\item if the aggregation function call is \gl{COUNT(*)}, then rewrite into
$\gl{COUNT}(g)$

\item otherwise, rewrite $f(e)$ into 

\[ f(\gl{SELECT VALUE}\ e'\ \gl{FROM}\ g\ \gl{AS}\ p) \]

\noindent where $e'$ is produced from $e$ as follows: Consider the variables
$v_1, \ldots, v_n$  that appear in $B^{in}_{\gl{GROUP}}$ (i.e., the variables
defined by the query's \gl{FROM} and \gl{LET} clauses) and are not grouping
attributes. Substitute each identifier $v_i$ (that does not stand for attribute
name) in $e$ with $p.v_i$.
\end{itemize}

\yannis{to self: Do we first dereference attribute names to variable.attribute
names and then rewrite the GROUP BY? Or vice versa? Or all together?}

\yannis{to self: Add ``database name" in query syntax under path}

\begin{example}
\label{xmpl:groupby-sql}

Consider again the query of Example~\ref{xmpl:groupby-avg-count}. It can be
written in an SQL compatible way as

\begin{tabbing}
\ \ \ \=\gl{SELECT }\=\gl{l.sensor AS sensor,}\\
\>\>\gl{AVG(l.co) AS avg, }\\
\>\>\gl{COUNT(*) AS count}\\
\>\gl{FROM logs AS l}\\
\>\gl{GROUP BY l.sensor}
\end{tabbing}
\end{example}

\begin{example}
\label{xmpl:group-by-nothing-sql}
The query of Example~\ref{xmpl:group-all-core} can be written in standard SQL
syntax as
\begin{tabbing}
\ \ \ \=\gl{SELECT  COUNT(g) AS largeco}\\
\>\gl{FROM logs AS l}\\
\>\gl{WHERE l.co > 1.5}
\end{tabbing}
\end{example}

Notice that SQL does not allow nested aggregate functions. Respectively, PartiQL
does not allow one to write queries that lack a \gl{GROUP AS} or \gl{GROUP ALL}
clause and have nested aggregate SQL functions.

\subsubsection{Designation of SQL aggregate functions}
\label{sec:SQL-aggregation-functions}

Each implementation will have a list of SQL aggregate functions, which are not
necessarily just the ones prescribed by the standard (\gl{COUNT}, \gl{SUM},
\gl{AVG}, etc). (Recall from Section~\ref{sec:implicit-group-variable} that SQL
aggregate functions do not use an explicit group variable.)

Furthermore, it is required that for each SQL aggregate function $f$, if an
implementation offers a corresponding core PartiQL aggregate function, the
PartiQL function is named \gl{COLL\_$f$}. For example, the core PartiQL
aggregate \gl{COLL\_AVG} corresponds to the SQL aggregate \gl{AVG}.
Nevertheless, it is possible that an implementation offers only \gl{COLL\_AVG}
or offers only \gl{AVG}. The semantic relationship between the SQL aggregate
function and the corresponding core PartiQL aggregate function is the one
explained in Section~\ref{sec:implicit-group-variable}: The SQL aggregate
functions do not input explicit group variables and, thus, their semantics are
explained by the reduction to the corresponding core PartiQL aggregate.

\subsubsection{Aliases from \gl{SELECT} clause}
\label{sec:select-aliases-groupby}

In SQL, a grouping expression may be an alias that is defined by the \gl{SELECT}
clause. For compatibility purposes, PartiQL adopts the same behavior. 

The query (S), which uses the \gl{SELECT}-defined alias feature, is syntactic
sugar for the query (C). Notice that the grouping expression $a$ is simply a
shorthand for $e$.

\begin{tabular}{@{}l@{~}l@{~}l@{~}l@{}}
(S) & \gl{SELECT ~} $\ldots\gl{,} e\ \gl{AS}\ a\gl{,} \ldots$    & \ \ \ (C)   & \gl{SELECT ~} $\ldots\gl{,} e\ \gl{AS}\ a \gl{,} \ldots$ \\
    & \gl{FROM ~~~} \ldots                          &    & \gl{FROM ~~~} \ldots \\
    & \gl{GROUP BY} $\ldots\gl{,} a\gl{,} \ldots$         &       & \gl{GROUP BY} $\ldots\gl{,} e\gl{,} \ldots$ \\
\end{tabular}

In the case that the grouping expression is a constant positive integer literal
$n$, then it stands for the $n$th attribute of the \gl{SELECT} clause. However,
this requires that the tuples produced by the \gl{SELECT} have schema and they
are ordered tuples. The relevant examples will be provided in the schema
section.

\begin{example}
\label{xmpl:select-aliases-groupby}
Consider the database 
\begin{tabbing}
\ \ \ \gt{people: \ob}\=\gt{\{'name': 'zoe',  'age': 10, 'tag': 'child'\},}\\
\>\gt{\{'name': 'zoe', 'age': 20, 'tag': 'adult'\},}\\
\>\gt{\{'name': 'bill', 'age': 30, 'tag': 'adult'\}}\\
\>\gt{\cb}
\end{tabbing}
The query
\begin{tabbing}
\ \ \ \=\gl{SELECT p.tag || ':' || p.name AS tagname, AVG(p.age) AS average}\\
\>\gl{FROM people AS p}\\
\>\gl{GROUP BY tagname}
\end{tabbing}
\noindent is equivalent to the query 
\begin{tabbing}
\ \ \ \=\gl{SELECT p.tag || ':' || p.name AS tagname, AVG(p.age) AS average}\\
\>\gl{FROM people AS p}\\
\>\gl{GROUP BY p.tag || ':' || p.name}
\end{tabbing}
Either query results into 
\begin{tabbing}
\ \ \ \gt{people: \ob}\=\gt{\{'tagname': 'child:zoe',  'average': 10\},}\\
\>\gt{\{'tagname': 'adult:zoe', 'average': 20\},}\\
\>\gt{\{'tagname': 'adult:bill', 'average': 30\}}\\
\>\gt{\cb}
\end{tabbing}
\end{example}
 
\subsection{Windowing cases simplified by the PartiQL grouping}
\label{sec:windows-by-grouping}

\begin{example}
\label{xmpl:windows-by-grouping}
Consider again a collection of sensor readings, this time with a timestamp. 

\begin{tabbing}
\ \ \ \=\gt{logs: [}\=\gt{\{'sensor':1, 'co':0.4, 'timestamp':04:05:06\},}\\
\>\>\gt{\{'sensor':1, 'co':0.2, 'timestamp':04:05:07\},}\\
\>\>\gt{\{'sensor':1, 'co':0.5, 'timestamp':04:05:10\},}\\
\>\>\gt{\{'sensor':2, 'co':0.3\}}\\
\>\>\gt{]}
\end{tabbing}

We look for the ``jump" readings that are more than 2x the previous reading at
the same sensor. The following query solves the problem using \gl{GROUP BY}.

\begin{tabbing}
\ \ \ \=\gl{SELECT }\=\gl{sensor AS sensor,}\\
                          \>\>\gl{(}\=\gl{WITH }\=\gl{orderedReadings}\\
                          \>\>\>\>\gl{AS (SELECT v FROM oneSensorsReadings v ORDER BY v.timestamp)}\\
                          \>\>\>\gl{SELECT r.co, r.timestamp }\\
                          \>\>\>\gl{FROM orderedReadings r AT p}\\
                          \>\>\>\gl{WHERE r.co > 2*orderedReadings[p-1].co}\\
                          \>\>\>\gl{ORDER BY p}\\
                          \>\>\>\gl{) AS jumpReadings}\\
       \>\gl{FROM logs l}\\
       \>\gl{GROUP BY l.sensor AS sensor GROUP AS oneSensorsReadings}
\end{tabbing}

The result is
\begin{tabbing}
\ \ \ \=\gt{\ob}\=\gt{\{'sensor':1, 'jumpReadings':[\{'co':0.4, 'timestamp':04:05:06\}]\},}\\
\>\>\gt{\{'sensor':2, 'jumpReadings':[]\}}\\
\>\>\gt{\cb}
\end{tabbing}

\end{example}
