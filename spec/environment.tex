\section{Queries, Environments and Binding Tuples}
\label{section:environment-and-sfw}
 
\begin{figure}[ht!]
\centering
\scriptsize
\begin{tabular}{|@{~}rc@{~}l@{~}|}
\hline
         1  & \multicolumn{2}{@{}l@{~}|}{\gn{query}} \\
         2  & \gp & \gn{sfw\_query} \\
         3  & \gd & \gn{expr\_query} \\ \hline
         4  & \multicolumn{2}{@{}l@{~}|}{\gn{sfw\_query}} \\
         5  & \gp & (\gt{WITH} \gn{query} \gt{AS} \gn{variable})? \\
         6  &     & \gn{select\_clause} \\ %FIXME (see Figure~\ref{figure:select:bnf}) \\
         7  &     & \gn{from\_clause} \\   %FIXME (see Figure~\ref{figure:from:bnf}) \\ 
         8  &     & (\gt{WHERE} \gn{expr\_query})? \\
         9  &     & (\gt{GROUP BY} \gn{expr\_query} (\gt{AS} \gn{variable})? \\ 
        10  &     & ~~~~(\gt{,} \gn{expr\_query} (\gt{AS} \gn{variable})?)*)? \\
        11  &     & ~~~~\gt{GROUP AS} \gn{variable} \\
        12  &     & (\gt{HAVING} \gn{expr\_query})? \\
        13  &     & ((\gt{OUTER})? (\gt{UNION}\gd\gt{INTERSECT}\gd\gt{EXCEPT}) \gt{ALL}? \gn{sfw\_query})? \\
        14  &     & ((\gt{ORDER BY} (\gn{expr\_query} (\gt{ASC}\gd\gt{DESC})? \gs{order\_spec}?  \\
        15  &     & ~~~~~~~~(\gt{,} \gn{expr\_query} (\gt{ASC}\gd\gt{DESC})? \gs{order\_spec}?)*)? ) \\
        16  &     & ~~~~| \gt{PRESERVE})? \\
        17  &     & (\gt{LIMIT} \gn{expr\_query})? \\
        18  &     & (\gt{OFFSET} \gn{expr\_query})? \\ \hline
        19  & \multicolumn{2}{@{}l@{~}|}{\gn{expr\_query}} \\
        20  & \gp & \gt{(} \gn{sfw\_query} \gt{)} \\
        21  & \gd & \gn{path\_expr} \\
        22  & \gd & \gn{function\_name} \gl{(} (\gn{expr\_query} (\gt{,} \gn{expr\_query})*)? \gl{)} \\
        23  & \gd & \gt{\{} (\gs{expr\_query}\gt{:}\gn{expr\_query} (\gt{,} \gs{expr\_query}\gt{:}\gn{expr\_query})*)? \gt{\}} \\
        24  & \gd & \gt{[} (\gn{expr\_query} (\gt{,} \gn{expr\_query})*)? \gt{]} \\
        25  & \gd & \gl{\ob} (\gn{expr\_query} (\gt{,} \gn{expr\_query})*)? \gl{\cb} \\
        26  & \gd & \gs{sql\_scalar\_expr} \\
        27  & \gd & \gs{value\_constant} \\ \hline
        28  & \multicolumn{2}{@{}l@{~}|}{\gn{path\_expr}} \\
        29  & \gp & \gs{variable} \\
%        30  & \gd & \gt{(} \gn{sfw\_query} \gt{)} \\
        30  & \gd & \gt{(} \gn{expr\_query} \gt{)} \\
        31  & \gd & \gn{path\_expr} \gt{.} \gs{attr\_name} \\
        32  & \gd & \gn{path\_expr} \gt{[} \gs{expr\_query} \gt{]} \\
        33  & \gd & \gn{path\_expr} \gt{.} \gl{*} \\
        34  & \gd & \gn{path\_expr} \gt{[} \gl{*} \gt{]} \\
\hline
\end{tabular} 
\caption{BNF Grammar for PartiQL Queries}
\label{figure:query:bnf}
\end{figure}

\newcommand{\linequery}[1]{%
    \IfEqCase*{#1}{%
    {sfw query}{(Figure~\ref{figure:query:bnf}, lines~4--18)}%
    {expression query}{lines~19--34}%
    {function invocation}{(Figure~\ref{figure:query:bnf}, line~22)}%
    {path expressions}{(Figure~\ref{figure:query:bnf}, lines~28--34)}%
    {tuple nav}{(Figure~\ref{figure:query:bnf}, lines~32--33)}%
    {array nav}{(Figure~\ref{figure:query:bnf}, line~32)}%
    {sfw subquery}{Figure~\ref{figure:query:bnf}, line~20}%
    {expression, sql}{(Figure~\ref{figure:query:bnf}, line~26)}%
    {expression, extensions}{(Figure~\ref{figure:query:bnf}, lines~21--25, 28--34)}%
    {constructors}{(Figure~\ref{figure:query:bnf}, lines~23--25)}%
    {group by}{(Figure~\ref{figure:query:bnf}, lines~9--11)}%
    }[\errmessage{Unable to ref #1 for query BNF}]%
}

PartiQL may be seen as a functional programming language with composable
semantics. Three aspects of the PartiQL syntax and semantics are characteristic
of its functional orientation: First, every (sub-)query and every
(sub-)expression input and output PartiQL data. Second, each clause of an SFW
query (\gt{SELECT}-\gt{FROM}-\gt{WHERE}) is itself a function. Third, every
(sub-)query evaluates within an \emph{environment} created by the database names
and the variables of the enclosing queries.

\subsection{Basics of PartiQL Syntax} 
\label{sec:syntax-basics}

A PartiQL query is either an \textit{SFW query} (i.e.
\gt{SELECT}-\gt{FROM}-\gt{WHERE}-$\ldots$, \linequery{sfw query} of the grammar
of Figure \ref{figure:query:bnf}) or an \textit{expression query} (also called
\textit{simple expression} in the rest, \linequery{expression query}) such as a
path expression \linequery{path expressions} or a function invocation. Unlike
SQL expressions, which are restricted to outputting scalar and null values,
PartiQL expressions output arbitrary PartiQL values, and are fully composable
within larger SFW queries and expressions. Indeed, PartiQL allows the top-level
query to also be an expression query, not just a SFW query as in SQL.

An PartiQL (sub)query is evaluated within an environment, which provides
variable bindings (as defined next). 

\subsection{Environments} 
\label{sec:environments-and-bindings}

\begin{figure}[ht!]
\centering
\begin{tabular}{|r|lrl|}
\hline
   1  & \gn{bind\_name}         & \gp   & \gn{global\_name} \\
   2  &                         & \gd   & \gn{variable} \\ \hline
   3  & \gn{qualified\_name}    & \gp   & \gn{identifier} (\gl{.} \gn{identifier})+ \\
   4  & \gn{variable\_name}     & \gp   & \gn{identifier} \\ \hline
   5  & \gn{identifier}         & \gp   & (\gl{\textdollar} | \gl{\_} | \gs{letter})
                                          (\gl{\textdollar} | \gl{\_} | \gs{letter} 
                                            | \gs{digit})* \\
   6  &                         & \gd   & \gl{"} \gs{quoted\_identifier\_body} \gl{"} \\
\hline
\end{tabular}
\caption{BNF Grammar for PartiQL Names}
\label{figure:names:bnf}
\end{figure}

\newcommand{\linenames}[1]{
    \IfEqCase*{#1}{
    {bind name}{(Figure~\ref{figure:names:bnf}, lines~4--18)}
    {qualified name}{(Figure~\ref{figure:names:bnf}, line~3)}
    {variable name}{(Figure~\ref{figure:names:bnf}, line~4)}
    {identifier}{(Figure~\ref{figure:names:bnf}, lines~5--6)}
    }[\errmessage{Unable to ref #1 for identifiers BNF}]
}

Each PartiQL (sub-)query and PartiQL (sub-)expression $q$ is evaluated within
the \textit{database environment} $\db$ created by the database names and the
\textit{variables environment} $\env$ created by the defined query variables.
The pair of these environments, $\benv$ is collectively called the
\textit{bindings environment}.

In either case, an environment is a \textit{binding tuple} $\langle x_1:v_1,
\ldots, x_n:v_n \rangle$, where each $x_i$ is a \textit{bind name}
\linenames{bind name} that is unique and binds to the PartiQL \textit{value}
$v_i$. The two distinct environments may also be thought of as \textit{global}
(the database object names) and \textit{local} (the lexically defined variables
in a particular scope of the query).

% TODO we need a better section around the types

Similarly, for a given $q$ at compile (i.e. planning) time, a \textit{database
type environment}, $\typedb$, and \textit{variables type environment},
$\typeenv$ are defined. The type environment is a \textit{binding tuple}
$\langle x_1:\type_1, \ldots, x_n:\type_n \rangle$, where each $x_i$ is a
\textit{name} that is unique and binds to the PartiQL \textit{type} $\type_i$.
For schema-less values, $\type$ can be considered the union of any possible type
(for which all operations are \textit{potentially} applicable). This is
discussed in more detail in Section~\ref{sec:schema}.

\textit{Qualified names} \linenames{qualified name} only ever appear in the
\textit{database environment}. Lexically defined \textit{variable names}
\linenames{variable name} are always just simple identifiers
\linenames{identifier}. For example, a relational database might define a compound name
\texttt{mydb.log}, where \texttt{mydb} is the schema (and not actually a value)
and \texttt{log} could be a table name within that schema. Note, that a
qualified name is distinct from a quoted identifier that contains a dot. Thus,
the qualified name \texttt{mydb.log} is distinct from the bind name
\texttt{"mydb.log"}.

\begin{example}
Let us assume that we evaluate the following query on the database of
Figure~\ref{figure:values:example-value}, whose top-level value is named
\gt{mydb.log}.

\begin{lstlisting}
SELECT x.resourceId
FROM mydb.log.configurationItems x
\end{lstlisting}

The query is evaluated within the database environment 

\[\db = \langle \gt{mydb.log}:\gt{\{ 'fileversion':'1.0', 'configurationItems':}\ldots \gt{\}} \rangle \]

\noindent and the variables environment $\env_1 = \langle \rangle$. Notice the
database environment $\db$ has a single name/value pair, which corresponds to
the only name (\gt{mydb.log}) of the database of
Figure~\ref{figure:values:example-value}. The variables environment has no
name/value pair because the above query is not a subquery of a larger query. 

Next, consider the subexpression \gt{x.resourceId} of the example's query. This
subexpression will, generally, be evaluated many times - once for each \gt{x}.
Technically, each time it is evaluated within the same database environment
$\db$ and within a variables environment $\env_2 = \langle \gt{x}:\ldots
\rangle$, i.e., a variables environment that defines the variable \gt{x}. 
\end{example}

\noindent \textbf{Remark on relationship of binding tuples to PartiQL tuples}
A binding tuple is similar to a PartiQL tuple, if you think of the bind names as
attribute names. The characterization ``binding'' pertains to its use in the
semantics (e.g. an association of names to types) and the fact that qualified
names are not reified in the PartiQL data model, and we have a representation.
As we will see collections of binding tuples will be homogenous, i.e., they will
all have the same ``attribute'' names. Also important, is that when we represent
binding tuples we explicitly represent a variable with a \MISSING\
value, as opposed to omitting it because the lack of a variable name is distinct
from a variable whose value is \MISSING. For example, we write $\langle
\gt{x}:\gt{1}, \gt{y}:\MISSING \rangle$, instead of $\langle \gt{x}:\gt{1}
\rangle$.$\bullet$

\noindent \textbf{Evaluation in environment} The notation $\benv \vdash q
\rightarrow v$ denotes that the PartiQL query $q$ evaluates to the value $v$
when evaluated within the database environment $\db$ and the variables
environment $\env$, i.e. when every variable of $q$ is instantiated by its
binding in $\env$ and each database name is instantiated to its value in $\db$.
For example, consider the query \gt{x + y / 2}, the database environment $\db =
\langle \gt{x}:\gt{5} \rangle$ and the variables environment $\env = \langle
\gt{y}:\gt{3} \rangle$. Then $\db, \env \vdash \gt{(x + y)/2} \rightarrow
\gt{5+3/2} \rightarrow \gt{4}$.

\subsection{The semantics of each clause of an SFW query explained as
input and output of binding tuples}
\label{sec:clause-semantics}

The semantics of PartiQL are shorter than the semantics of SQL itself---despite
being backwards compatible with SQL. The key reason is that the semantics of
each clause of an SFW query 
in the PartiQL core can be understood in isolation from the other
clauses. A clause is simply a function that inputs and outputs binding tuples.
Thus the specifics of how the binding tuples of a query and of its subqueries
are produced are a central part of the semantics. At a high level (which will be
elaborated upon later) the construction of binding environments proceeds as
follows:

\begin{compact_enum} 
\item When a query is submitted to a database, it is evaluated in an empty
variables environment $\env = \langle \rangle$.

\item The \from\ clause of a SFW query produces new environments by
concatenating bindings of the \gl{FROM} variables to the environment of its SFW
query, as explained below. 

The subqueries that appear in the \gl{WHERE}, \gl{SELECT}, etc clauses are
evaluated in these new environments. The optional \gl{GROUP BY} clause also
produces additional variable bindings, as explained in
Section~\ref{section:group-by}.

\end{compact_enum}

\begin{figure}
\[ 
\begin{array}{l}
\db: \langle \gt{mydb.r}:\gt{[3, 'x' ]}, \gt{mydb.s}:\gt{\ob \{'a':1, 'b':2\}, \{'a':3\} \cb} \rangle \\
\env: \langle \rangle
\end{array}
\]
\begin{tabbing}
\gt{FROM }\=\gt{mydb.r AS x, mydb.s AS y}\\
\>\ $B^{out}_{\gt{FROM}} = B^{in}_{\gt{WHERE}} = \ob$\=$\langle \gt{x:3, y:\{'a':1, 'b':2\}} \rangle$\\
\>\>$\langle \gt{x:3, y:\{'a':3\}} \rangle$\\
\>\>$\langle \gt{x:'x', y:\{'a':1, 'b':2\}} \rangle$\\
\>\>$\langle \gt{x:'x', y:\{'a':3\}} \rangle$\\
\>\>\!\!$\cb$\\
\gt{WHERE x > y.b}\\
\>\ $B^{out}_{\gt{WHERE}} = B^{in}_{\gt{SELECT}} = \ob\langle \gt{x:3, y:\{'a':1, 'b':2\}} \rangle$ \cb\\
\gt{SELECT x AS foo, y.a AS bar}\\
\>Result: $\ob \gt{\{foo:3, bar:1\}} \cb$
\end{tabbing}
\caption{An Example SFW Query with Flow of Binding Tuples}
\label{fig:xmpl:sfw-bindings}
\end{figure}

\paragraph{SFW query clauses as operators that input/output binding tuples}
Similar to SQL semantics, the clauses of an PartiQL SFW query are evaluated in
the following order: \gl{WITH}, \gl{FROM}, \gl{LET}, \gl{WHERE}, \gl{GROUP BY},
\gl{HAVING}, \gl{LETTING} (which is special to PartiQL), \gl{ORDER BY},
\gl{LIMIT / OFFSET} and \gl{SELECT} (or \gl{SELECT VALUE} or \gl{PIVOT}, which
are both special to ion PartiQL).
\footnote{PartiQL also supports a syntax improvement where \gl{SELECT}
is optionally written as the last clause since, anyway, that's the proper
way to read an SQL query.}

% Notice that PartiQL SFW queries may also have a \gl{LET} clause, between
% the \gl{FROM} and \gl{WHERE} clauses and a \gl{LETTING} clause between
% the \gl{HAVING} and \gl{ORDER BY} clauses. 

\yannis{to do: add LET and LETTING in syntax.}

Using the example of Figure~\ref{fig:xmpl:sfw-bindings}, we illustrate how the
clauses of an SFW query input and output binding tuples. In the
Figure~\ref{fig:xmpl:sfw-bindings}, the \gl{FROM}, \gl{WHERE} and \gl{SELECT}
clauses of the example query are displayed apart from each other so that the
example can also illustrate the binding tuples that flow from the one clause to
the next. 

The query is evaluated within the bindings environment $\benv$ shown at the top
of Figure~\ref{fig:xmpl:sfw-bindings}. Consequently, the \gl{FROM} clause is
evaluated in the same environment. Thereafter the \gl{FROM} clause outputs the
bag of binding tuples $B^{out}_{\gl{FROM}}$, which has four binding tuples in
the example. In each binding tuple of $B^{out}_{\gl{FROM}}$, each variable of
the \gl{FROM} clause is bound to a value. There are no restrictions that a
variable binds to homogenous values across binding tuples. In the example,
\gl{x} binds to two values that are heterogeneous: some bindings of \gl{x} bind
to a number, while others to a string. It would also be possible that a variable
binds to, say, a scalar in one binding, while the same variable binds to a
complex value in another binding.

Each subsequent clause inputs a bag of binding tuples, evaluates the constituent
expressions of the clause (which may themselves contain nested SFW queries), and
outputs a bag of binding tuples that is in turn input by the next clause. For
instance, the \gl{WHERE} clause inputs the bag of binding tuples that have been
output by the \gl{FROM} clause ($B^{out}_{\gl{FROM}} = B^{in}_{\gl{WHERE}}$),
and outputs the subset thereof that satisfies the condition expression of the
\gl{WHERE} clause. This subset is the $B^{out}_{\gl{WHERE}} =
B^{in}_{\gl{SELECT}}$. 

In particular, the \gl{WHERE}'s condition is evaluated once for each input
binding tuple $b$ in $B^{in}_{\gl{WHERE}}$. In general, each evaluation is done
within the bindings environment $(\db, \env \| b)$, i.e., the concatenation of
the binding tuple $\env$ (where $\env$ is the binding environment of the SFW
query) with the binding tuple $b$ that has the variables of the \gl{FROM}
clause. In the particular example $\env \| b$ is simply $b$ since $\env =
\langle \rangle$. The condition \gt{x > y.b} is evaluated once for each of the
four input binding tuples of $B^{in}_{\gt{WHERE}}$. The variables environment of
the first evaluation is:

\[ 
\env = \langle \gt{x}:\gt{3}, \gt{y}:\gt{\{'a':1, 'b':2\}} \rangle 
\] 

\noindent The \gl{WHERE} condition evaluates to \gt{true} for the first
binding tuple of  $B^{in}_{\gt{WHERE}}$, since 

\[ \benv \vdash \gt{x > y.b} \rightarrow \gt{3 > \{'a':1, 'b':2\}.b} \rightarrow \gt{true} \]

\noindent Thus the first binding tuple of $B^{in}_{\gt{WHERE}}$ is output from
the \gl{WHERE} and is input to \gl{SELECT}.

The pattern of ``input bag of binding tuples, evaluate constituent expressions,
output bag of binding tuples'' has a few exceptions: First, the \gt{ORDER BY}
clause inputs a bag of binding tuples and outputs an array of binding tuples.
Second, a \gt{LIMIT}/\gt{OFFSET} clause need not evaluate its constituent
expression for each input binding tuple. For example a ``\gt{LIMIT} \texttt{10}"
clause that inputs an array with 100 binding tuples need not access binding
tuples 11-100. 

Finally, the \gt{SELECT} clause is responsible for converting from binding
tuples to collections of arbitrary PartiQL elements. The \gt{SELECT} inputs a
bag (or array, if \gl{ORDER BY} is present) of binding tuples, and outputs
the SFW query's result, which is a bag (resp. array) with exactly one element
for each input binding tuple. In the example, the \gl{SELECT} expressions \gt{x}
and \gt{y.a} are evaluated once for each of the input binding tuples of
$B^{in}_{\gl{SELECT}}$, which in this example happen to be just one binding
tuple.

Finally, notice that the above discussion of SFW queries did not capture the set
operators \gl{UNION}, \gl{INTERSECT} and \gl{EXCEPT}. As is the case with SQL
semantics too, the coordination of \gl{ORDER BY} with the set operators requires
attention. \eat{FIXME, as discussed in Section~\ref{section:order-by}
\ref{sec:order-with-set}.}

\paragraph{PartiQL clauses as operators} 
In summary, each clause of PartiQL is an operator that inputs/outputs binding
tuples. As such, we can (and will) present the semantics of each clause
separately from the semantics of the other clauses. This is not the case in SQL:
Notably, in the presence of aggregation functions the \gl{SELECT}, \gl{HAVING}
and \gl{WHERE} cannot be interpreted in isolation; they can only be interpreted
along with the \gl{GROUP BY} clause. 

\subsection{Scoping Rules of Variables} 
\label{sec:scoping-variables}

As in any programming language, the PartiQL semantics have to deal with issues
of variable scope. For example, how are references to \gt{x} resolved in the
following query:

\begin{lstlisting}
SELECT x.a AS a
FROM db1 AS x
WHERE x.b IN (SELECT x.c FROM db2 AS x)
\end{lstlisting}

Since this is an SQL query and PartiQL is backwards compatible to SQL, it is
easy to tell that the \gt{x} in \gt{x.c} resolves to the variable \gt{x} defined
by the inner query's \gl{FROM} clause.

Technically, this scoping rule is captured by the following handling of binding
tuples. The inner \gl{FROM} clause is evaluated with a variables environment
$\env = \langle \gt{x}: \ldots\rangle$; its \gt{x} is the one defined by the
outer \gl{FROM}. Then the inner \gl{FROM} clause outputs a binding $b = \langle
\gt{x}: \ldots\rangle$; this \gt{x} is defined by thinner \gl{FROM}. Then the
\gt{x.c} is evaluated in the concatenation $\env \| b$ and because \gt{x}
appears in both $\env$ and $b$, the concatenation keeps only the \gt{x} of its
right argument. Essentially by putting $b$ as the right argument of the
concatenation, the semantics indicate that the variables of $b$ have precedence
over synonymous variables in the left argument (which was the $\env$).

Generally, given two binding tuples $b$ and $b'$, their concatenation is a
binding tuple, denoted as $b \| b'$, that has the variable bindings of both $b$
and $b'$. This creates the possibility that both $b$ and $b'$ have the same
variable $x$. In this case, the concatenation $b || b'$ will have the $b'.x$ and
its value; it will not have the $b.x$ and its value.

Note, the above does not resolve scoping issues resulting from conflicts between
the database environment and the variables environment. We resolve these
conflicts by explicit rules.
